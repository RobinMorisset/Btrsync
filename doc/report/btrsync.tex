\documentclass[11pt]{llncs}

\def\makeitbig{%
\setlength{\textwidth}{15.9cm}%
\setlength{\oddsidemargin}{.01cm}%
\setlength{\evensidemargin}{.01cm}%
\setlength{\textheight}{21.5cm}%
\setlength{\topmargin}{-.25cm}%
\setlength{\headheight}{.7cm}%
\leftmargini 20pt     \leftmarginii 20pt%
\leftmarginiii 20pt   \leftmarginiv 20pt%
\leftmarginv 12pt     \leftmarginvi 12pt%
\pagestyle{myheadings}}%

\makeitbig

\usepackage{algorithmicx}
\usepackage[english]{babel}
\usepackage[utf8]{inputenc}
\usepackage{amsmath, amsfonts, amssymb, graphicx, rotating, epsfig}
\usepackage{verbatim}
\usepackage{algorithm}
\usepackage[noend]{algpseudocode}
\usepackage{url}
\usepackage{tikz}
\usepackage{tabularx,multirow}
\usepackage{xspace}

\newcommand*\Let[2]{\State #1 $\gets$ #2}
%\algrenewcommand\alglinenumber[1]{
%    {\sf\footnotesize\addfontfeatures{Colour=888888,Numbers=Monospaced}#1}}


\newcommand{\sig}{{\sf $($Gene\-rate, Sign, Verify$)$} }
\newcommand{\ignore}[1]{}
\newcommand{\btr}{{\tt btrsync}}
\newcommand{\rsy}{{\tt rsync}}
\newcommand{\abs}[1]{\left|#1\right|}
\newcommand{\Cov}[0]{\mbox{Cov}}
\newcommand{\Var}[0]{\mbox{Var}}
\newcommand{\xor}[0]{\oplus}
\newcommand{\rmu}[0]{\mbox{RM}}
\newcommand{\Prob}[1]{{\Pr\left[\,{#1}\,\right]}}
\newcommand{\EE}[1]{{\mathbb{E}\left[{#1}\right]}}
\newcommand{\Oapp}{\ensuremath{\tilde{O}}}

\newcommand{\btrsync}{\texttt{btrsync}\xspace}
\newcommand{\rsync}{\texttt{rsync}\xspace}

\newcommand{\comm}[1]{\marginpar{%
\vskip-\baselineskip %raise the marginpar a bit
\raggedright\footnotesize
\itshape\hrule\smallskip#1\par\smallskip\hrule}}

\usepackage{hyperref}

\begin{document}

\title{When File Synchronization Meets Number Theory}

\author{Antoine Amarilli \and Fabrice Ben Hamouda \and Florian Bourse \and\\
Robin Morisset \and David Naccache \and Pablo Rauzy}

\institute{
\'{E}cole normale sup\'{e}rieure, D\'{e}partement d'informatique \\
   45, rue d'Ulm, {\sc f}-75230, Paris Cedex 05, France.\\
   \email{surname.name@ens.fr} (except for \email{fabrice.ben.hamouda@ens.fr})
}

\maketitle

\begin{abstract}
This work revisits {\sl set reconciliation}, a problem consisting in synchronizing two multisets whiles minimizing communication.
We propose a new number theoretic reconciliation protocol called ``Divide \& Factor''. In terms of asymptotic  transmission complexity, Divide \& Factor is comparable to prior proposals (that anyhow reached optimality). Nonetheless, the new protocols offer interesting parameter trade-offs resulting in experimentally measured {\sl constant}-factor transmission gains over the popular software \rsync.\smallskip

Reconciliation experiments show that the new protocol usually transmits less data than \rsync\ but requires lengthier calculations.\smallskip
\end{abstract}

\section{Introduction}

This work revisits {\sl set reconciliation}, a problem consisting in synchronizing two multisets while minimizing communication complexity. Set reconciliation is necessary in many practical situations, the most typical of which is certainly incremental information backup.\smallskip

Set reconciliation has already several efficient and elegant solutions. For instance, \cite{PSRec} presents a particularly interesting reconciliation protocol whose computational and communication complexities are linear in the number of differences between the reconciled multisets.\smallskip

We refer the reader to \cite{PSRec,Mins1,Whats} (to quote a few references) for more on the problem's history and its existing solutions.\smallskip

This article proposes a new reconciliation protocol based on number theory. In terms of asymptotic  transmission complexity, the proposed procedure is comparable to prior proposals (that anyhow reached optimality). Nonetheless, the new protocols offer interesting parameter trade-offs resulting in experimentally measured {\sl constant}-factor gains over the popular software \rsync.\smallskip

Indeed, during most of our reconciliation experiments, the new protocol transmitted less bytes than \rsync\ but required lengthier calculations.\smallskip

Beyond these constant-factor gains, the mathematical ideas underlying the protocol seem new and interesting as such.

\section{``Divide \& Factor'' Set Reconciliation}

\subsection{Problem Definition and Notations}

Oscar possesses an old version of a directory $\mathfrak{D}$ that he wishes to update. Neil has the up-to-date version $\mathfrak{D}'$. The challenge faced by Oscar and Neil\footnote{Oscar and Neil will respectively stand for {\sl \underline{o}ld} and {\sl \underline{n}ew}.} is that of {\sl exchanging as little data as possible} during the synchronization process. In practice $\mathfrak{D}$ and $\mathfrak{D}'$ usually differ both in their files and in their tree structure.\smallskip

In tackling this problem we separate the {\sl ``what''} from the {\sl ``where''} and disregard the relative position of files in subdirectories\footnote{{\sl i.e.} view directories as multisets of files.}. Let $\mathfrak{F}$ and $\mathfrak{F}'$ denote the multisets of files contained in $\mathfrak{D}$ and $\mathfrak{D}'$. We denote $\mathfrak{F}=\{F_0,\ldots,F_{n}\}$ and $\mathfrak{F}'=\{F'_0,\ldots,F'_{n'}\}$.\smallskip

Let $\mbox{{\tt Hash}}(F)$ be a collision-resistant hash function\footnote{{\sl e.g.} SHA-1} where $F$ is a file. Denote by $\mbox{{\tt NextPrime}}(F)$ the first prime following $\mbox{{\tt Hash}}(F)$ and let $u$ denote {\tt NextPrime}'s output size in bits\footnote{We assume that the fact that $u$-bit strings contain only $u/\log u$ primes does not affect the collision-resistance properties of $\mbox{{\tt NextPrime}}$.}. Define the shorthand notations: $h_i=\mbox{{\tt NextPrime}}(F_i)$ and $h'_i=\mbox{{\tt NextPrime}}(F'_i)$.\smallskip

\comm{TODO(amarilli): use the uniform nextprime (discussion of relative costs with respect to (1.) hashing costs and (2.) finding the next prime costs) C'est quoi? (David).}

\subsection{Description of the Basic Exchanges}
\label{basic}

Let $t$ be the number of discrepancies between $\mathfrak{F}$ and $\mathfrak{F}'$ that Oscar wishes to learn, {\sl i.e.}:

$$t=\#\mathfrak{F}+\#\mathfrak{F}'-2 \#\left(\mathfrak{F} \bigcap \mathfrak{F}'\right)=\#\left(\mathfrak{F}\bigcup\mathfrak{F}'\right)-\#\left(\mathfrak{F}\bigcap\mathfrak{F}'\right)$$

We generate a prime $p$ such that:

\begin{equation}
\label{equp}
2^{2ut+1} \leq p < 2^{2ut+2}
\end{equation}

Given $\mathfrak{F}$, Oscar generates and sends to Neil the redundancy:

$$
c=\prod_{F_i\in \mathfrak{F}} \mbox{{\tt NextPrime}}(F_i)=\prod_{i=1}^n h_i \bmod p
$$

Neil computes:\smallskip

$$c'=\prod_{F'_i\in \mathfrak{F'}} \mbox{{\tt NextPrime}}(F'_i)=\prod_{i=1}^{n'} h'_i \bmod p{~~~\mbox{and}~~~}s=\frac{c'}{c} \bmod p$$

Using \cite{vallee} the integer $s$ can be written as:
$$s=\frac{a}{b} \bmod p{\mbox{~where the~}G_i\mbox{~denote files and~}}
\left\{
\begin{array}{lcr}
a & =&  \prod\limits_{G_i \in \mathfrak{F}'\wedge G_i \not\in\mathfrak{F}} \mbox{{\tt NextPrime}}(G_i) \\
\\
b & = & \prod\limits_{G_i \not\in\mathfrak{F}' \wedge G_i \in\mathfrak{F}} \mbox{{\tt NextPrime}}(G_i)
\end{array}
\right.
$$

Note that since $\mathfrak{F}$ and $\mathfrak{F}'$ differ by at most $t$ elements, $a$ and $b$ are strictly lesser than $2^{ut}$. Theorem \ref{theo} (see \cite{cryptorational}) guarantees $a$ and $b$ can be efficiently recovered from $s$ (A problem  known as the {\sl Rational Number Reconstruction} \cite{pan2004rational,wang2003acceleration}). This is typically done using Gauss' algorithm for finding the shortest vector in a bi-dimensional lattice \cite{vallee}.

\begin{theorem}
\label{theo}
Let $a,b \in {\mathbb Z}$ such that $-A \leq a \leq A$ and $0<b \leq B$. Let $p>2AB$ be a prime and $s=a b^{-1} \mod p$.
Then $a,b$ can be recovered from $A,B,s,p$ in polynomial time.
\end{theorem}

Taking $A=B=2^{ut}-1$, (\ref{equp}) implies that $2AB<p$. Moreover, $0 \leq a \leq A$ and $0 <b \leq B$. Thus Oscar can
recover $a$ and $b$ from $s$ in polynomial time. By testing the divisibility of $a$ and $b$ by the $h_i$ and the $h'_i$, Neil and Oscar can
deterministically identify the discrepancies between $\mathfrak{F}$ and $\mathfrak{F}'$ and settle them.\smallskip

Formally, this is done as follows:\smallskip

\begin{center}
\begin{tabular}{|lcl|}\hline
~~{\bf Oscar}                       &                                                      &   {\bf Neil}~\\
                                   &~~{{\LARGE $\stackrel{c}{\longrightarrow}$}}~~        &   \\
                                   &                                                      &computes $a,b$~\\
                                   &                                                      &if $a$ doesn't factor into a product of $h'_i$s~\\
                                   &                                                      &~~~~~~then output $\bot$ and halt~\\
                                   &                                                      &~~~~~~else let $\mathfrak{S}=\{F'_i \mbox{~s.t.~} a \bmod h'_i =0\}$~~\\
                                   &~~{\LARGE $\stackrel{\mathfrak{S},b}{\longleftarrow}$}&\\
~~deletes files s.t. $b \bmod h_i =0$&                                                      &\\
~~adds $\mathfrak{S}$ to the disk    &                                                      &\\\hline
\end{tabular}
\end{center}

As we have just seen, the ``output $\bot$ and halt'' protocol interruption should actually never occur if bounds on parameters are respected. However, a file synchronization procedure that works {\sl only} for a limited number of differences is not really useful in practice. Section \ref{insuf} explains how to extend the protocol even when the number of differences exceeds $t$, the informational capacity of the modulus $p$.

\subsection{The Case of Insufficient Information}
\label{insuf}
To extend the protocol to an arbitrary $t$, Oscar and Neil agree on an infinite set of primes $p_1,p_2,\ldots$ As long as the protocol fails, Neil will keep accumulating information about the difference between $\mathfrak{F}$ and $\mathfrak{F}'$ as shown in appendix A. Note that no information is lost and that the transmitted modular knowledge about the difference adds-up until it reaches a threshold sufficient to reconciliate $\mathfrak{F}$ and $\mathfrak{F}'$.\smallskip

\section{Efficiency Considerations}

This section explores two strategies to reduce the size of $p$ and hence improve transmission by {\sl constant factors} (from an asymptotic communication standpoint, nothing can be done as the protocol already transmits information proportional to $t$, the difference to settle).

\subsection{Probabilistic Decoding: Reducing $p$}

Generate a prime $p$ about twice shorter than the $p$ recommended in section \ref{basic}, namely:
\begin{equation}
\label{eqnewp}
2^{ut+w-1}<p \leq 2^{ut+w}
\end{equation}

where $w \geq 1$ is some small integer (say $w=50$). Let $\eta=\max(n,n')$. The new redundancy $c$ is calculated as previously and is hence also approximately twice smaller. Namely:

$$s=\frac{a}{b} \bmod p \mbox{~and~}
\left\{
\begin{array}{lcr}
a & =&  \prod\limits_{G_i \in \mathfrak{F}'\wedge G_i \not\in\mathfrak{F}} \mbox{{\tt NextPrime}}(G_i) \\
\\
b & = & \prod\limits_{G_i \not\in\mathfrak{F}' \wedge G_i \in\mathfrak{F}} \mbox{{\tt NextPrime}}(G_i)
\end{array}
\right.
$$

and since there are at most $t$ differences, we must have:
\begin{equation}
\label{eqab}
a b \leq 2^{ut}
\end{equation}

The difference with respect to section \ref{basic} is that we do not have a fixed bound for $a$ and $b$ anymore; equation (\ref{eqab}) only provides a bound for the {\sl product} $a b$. Therefore, we define a sequence of at most $\lceil ut/w \rceil+1$ couples of bounds:

$$\left(A_i,B_i\right)=\left(2^{wi},\left\lfloor \frac{p-1}{2^{wi+1}} \right\rfloor\right)\mbox{~~where~~}B_i>1\mbox{~~and~~}\forall i>0,~2 A_i B_i<p$$

Equations (\ref{eqnewp}) and (\ref{eqab}) imply that there must exist at least one index $i$ such that $0 \leq a \leq A_i$ and $0 <b \leq B_i$. Then using Theorem \ref{theo}, given $(A_i,B_i,p,s)$ one can recover $(a,b)$, and hence the difference between $\mathfrak{F}$ and $\mathfrak{F}'$.\smallskip

The problem is that (by opposition to section \ref{basic}) we have no guarantee that such an $(a,b)$ is unique. Namely, we could (in theory) stumble over an $(a',b')\neq (a,b)$ satisfying (\ref{eqab}) for some index $i' \neq i$. We expect this to happen with negligible probability (that we do not try to estimate here) when $w$ is large enough, but this makes the modified protocol heuristic only.\smallskip

\subsection{The File Laundry: Reducing $u$}

What happens if we brutally shorten $u$ in the basic Divide \& Factor protocol?\smallskip

As expected by the birthday paradox, we should start seeing collisions. Let us analyze the statistics governing the appearance of collisions.

Consider $\mbox{{\tt Hash}}$ as a random function from $\{0,1\}^*$ to $\{0,\dots,2^u-1\}$. Let $X_i$ be the random variable:

$$
X_i =
\left\{
\begin{array}{lcl}
1 & ~~~~&  \mbox{if file $F_i$ collides with another file.}\\
\\
0 & ~~~~&  \mbox{otherwise.}
\end{array}
\right.
$$

Clearly, we have $\Prob{X_i = 1} \le \frac{\eta -1}{2^u}$.
The average number of colliding files is hence:
\[ \EE{\sum_{i=0}^{\eta-1} X_i} \le \sum_{i=0}^{\eta-1} \frac{\eta -1}{2^u} = \frac{\eta (\eta - 1)}{2^u} \]

For instance, for $\eta=10^6$ files and 32-bit digests, the expected number of colliding files is less than $233$.\smallskip

However, it is important to note that a collision can only yield a {\sl false positive}, and never a {\sl false negative}. In other words, while a collision may obliviate a difference\footnote{{\sl e.g.} make the parties blind to the difference between {\tt index.htm} and {\tt iexplore.exe}.} a collision will never create a nonexistent difference {\sl ex nihilo}.\smallskip

Thus, it suffices to replace $\mbox{{\tt Hash}}(F)$ by a chopped $\hbar_{k,u}(F)=\mbox{{\tt MAC}}_k(F) \bmod 2^u$ to quickly filter-out file differences by repeating the protocol for $k=1,2,\ldots$ At each iteration the parties will detect new files and new deletions, fix these and ``launder'' again the remaining multisets.\smallskip

To understand why, under the assumption that {\tt MAC}s are random and independent, the probability that a stubborn file persists colliding decreases exponentially with the number of iterations $k$, assume that $\eta$ remains invariant between iterations and define the following random variables:\smallskip

$$
\begin{array}{rcl}
X^{\ell}_i & = &
\left\{
\begin{array}{lcl}
1 & ~~~~&  \mbox{if file $F_i$ collides with another file during iteration $\ell$.}\\
\\
0 & ~~~~&  \mbox{otherwise.}
\end{array}
\right.\\
\\
Y_i = \prod_{\ell=1}^k X^{\ell}_i & = &
\left\{
\begin{array}{lcl}
1 & ~~~~&  \mbox{if file $F_i$ collides with another file during the $k$ first protocol iterations.}\\
\\
0 & ~~~~&  \mbox{otherwise.}
\end{array}
\right.
\end{array}$$

By independence, we have:

 \[ \Prob{Y_i = 1} = \prod_{\ell=1}^k \Prob{X^{\ell}_i = 1} = \Prob{X^1_i = 1} \dots \Prob{X^k_i = 1} \le \left( \frac{\eta -1}{2^u} \right)^k \]
Therefore the average number of colliding files is:
\[
 \EE{\sum_{i=0}^{\eta-1} Y_i} \le \sum_{i=0}^{\eta-1} \left( \frac{\eta -1}{2^u} \right)^k =  \eta \left(\frac{\eta - 1}{2^u}\right)^k
\]

And the probability that at least one false positive will survive $k$ rounds is:

\[
\epsilon_k \le \eta \left(\frac{\eta - 1}{2^u}\right)^k
\]

For the previously considered instance\footnote{$\eta=10^6$,$u=32$.} we get $\epsilon_2 \le 5.43\%$ and $\epsilon_3 \le 2 \cdot 10^{-3}\%$.

\subsubsection{A more refined (but somewhat technical) analysis}

As mentioned previously, the parties can remove the files confirmed as different during iteration $k$ and work during iteration $k+1$ only with common and colliding files. Now, the only collisions that can fool round $k$, are the collisions of a file-pairs $(F_i,F_j)$ such that $F_i$ and $F_j$ have both already collided during {\sl all the previous iterations}\footnote{Note that we \underline{do not} require that $F_i$ and $F_j$ repeatedly collide {\sl which each other}. {\sl e.g.} we may witness during the first round $\hbar_{1,u}(F_1)=\hbar_{1,u}(F_2)$ and $\hbar_{1,u}(F_3)=\hbar_{1,u}(F_4)$ while during the second round $\hbar_{2,u}(F_1)=\hbar_{2,u}(F_4)$ and $\hbar_{2,u}(F_2)=\hbar_{2,u}(F_3)$.}. We call such collisions ``masquerade balls''. Define the random variable:

$$
Z^\ell_i =
\left\{
\begin{array}{lcl}
1 & ~~~~&  \mbox{if $F_i$ participated in masquerade balls during all $\ell$ first protocol iterations.}\\
\\
0 & ~~~~&  \mbox{otherwise.}
\end{array}
\right.
$$

\comm{Inclure ici image du bal masque}

\comm{Fabrice tu utilises la notation non definie $X^k_{i,j}$ peux-tu la preciser stp?\smallskip}

Set $Z^0_i = 1$ and write $p_\ell = \Prob{Z^{\ell-1}_{i} = 1 \text{ and } Z^{\ell-1}_{j} = 1} $ for all $\ell$ and $i \neq j$.
For $k \ge 1$, we have:
\begin{align*}
\Prob{Z^k_i=1} &= \Prob{\exists j\neq i \text{, } X^k_{i,j} = 1 \text{, } Z^{k-1}_{i} = 1  \text{ and } Z^{\ell-1}_{j} = 1}  \\
&\le \sum_{j=0, j\neq i}^{\eta-1} \Prob{X^{k-1}_{i,j} = 1} \Prob{Z^{k-1}_{i} = 1 \text{ and } Z^{k-1}_{j} = 1}  \\
&\le \frac{\eta-1}{2^u} p_{k-1}
\end{align*}
Furthermore $p_0 = 1$ and
\begin{align*}
p_\ell &= \Prob{X^{\ell}_0 = X^{\ell}_1 \text{, } Z^{\ell}_{0} = 1 \text{ and } Z^{\ell}_{1} = 1}
  + \Prob{X^{\ell}_0 \neq X^{\ell}_1 \text{, } Z^{\ell}_{0} = 1 \text{ and } Z^{\ell}_{1} = 1} \\
&\le \Prob{X^{\ell}_0 = X^{\ell}_1 \text{, } Z^{\ell-1}_{0} = 1 \text{ and } Z^{\ell-1}_{1} = 1} \\
  &\quad+ \sum_{i \ge 2, j \ge 2} \Prob{X^\ell_{0,i} = 1 \text{, } X^\ell_{1,j} = 1 \text{, } Z^{\ell-1}_{0} = 1 \text{ and } Z^{\ell-1}_{1} = 1} \\
&= \Prob{X^{\ell}_0 = X^{\ell}_1} \Prob{Z^{\ell-1}_{0} = 1 \text{ and } Z^{\ell-1}_{1} = 1} \\
  &\quad+ \sum_{i \ge 2, j \ge 2} \Prob{X^\ell_{0,i} = 1} \Prob{X^\ell_{1,j} = 1} \Prob{Z^{\ell-1}_{0} = 1 \text{ and } Z^{\ell-1}_{1} = 1} \\
&\le \frac{1}{2^u} p_{\ell-1} + \frac{(\eta-2)^2}{2^{2u}} p_{\ell-1} = p_{\ell-1}\left(\frac{1}{2^u}  + \frac{(\eta-2)^2}{2^{2u}}\right)
\end{align*}
hence:
\[ p_\ell \le \left( \frac{1}{2^u} + \frac{(\eta-2)^2}{2^{2u}} \right)^\ell, \]
and
\[ \Prob{Z^\ell_i=1} \le \left( \frac{1}{2^u} + \frac{(\eta-2)^2}{2^{2u}} \right)^{k-1} \]
And finally, the survival probability of at least one false positive after $k$ iterations satisfies:
\[
\epsilon'_k \le \frac{\eta(\eta-1)}{2^u} \left( \frac{1}{2^u} + \frac{(\eta-2)^2}{2^{2u}} \right)^{k-1}
\]

For $(\eta=10^6,u=32,k=2)$, we get $\epsilon'_2 \le 0.013\%$.\smallskip

\subsubsection{How to select $u$?}

For a fixed $k$, $\epsilon'_k$ decreases as $u$ grows. For a fixed $u$, $\epsilon'_k$ also decreases as $k$ grows. Transmission, however, grows with both $u$ (bigger digests) and $k$ (more iterations). We write for the sake of clarity: $\epsilon'_k = \epsilon'_{k,u,\eta}$.\smallskip

Fix $\eta$. Note that the number of bits transmitted per iteration ($\simeq 3ut$), is proportional to $u$. This yields an expected transmission complexity bounded by a quantity $T_{u,\eta}$ such that:\smallskip

\[T_{u,\eta} \propto u \sum_{k=1}^{\infty} k \cdot \epsilon'_{k,u,\eta}=
\frac{u \eta\left(\eta-1\right)}{2^u} \sum_{k=1}^{\infty} k \left( \frac{1}{2^u} + \frac{\left(\eta-2\right)^2}{2^{2u}} \right)^{k-1}=
\frac{u \eta\left(\eta-1\right) 8^u}{\left(2^u-4^u+\left(\eta-2\right)^2\right)^2}\]

Dropping the proportionality factor $\eta\left(\eta-1\right)$, neglecting $2^u \ll 2^{2u}$ and approximating $(\eta-2)\simeq\eta$, we can optimize the function:

\[
\phi_\eta(u)=\frac{u \cdot 8^u}{\left(4^u-\eta^2\right)^2}
\]

$\phi_{10^6}(u)$ admits an optimum for $u=19$.

\subsubsection{Note:} The previous analysis is a rough approximation, in particular:

\begin{itemize}
\item We consider $u$-bit prime digests while $u$-bit strings contain only about $2^u/u$ primes.\smallskip

\item In all our probability calculations $\eta$ can be replaced by the total number of differences $t$. It is reasonable to assume that in most {\sl practical} settings $t \ll \eta$, but extreme instances where $t\sim\eta$ can sometimes be encountered as well.\smallskip

\item We used a fixed $u$ in all rounds. Nothing forbids using a different $u_k$ at each iteration\footnote{...or even fine-tuning the $u_k$s adaptively, as a function of the laundry's effect on the progressively reconciliated multisets.}.

\item Our analysis treated $t$ as a constant, but large $t$ values increase $p$ and hence the number of potential files detected as different per iteration - an effect disregarded in our analysis.
\end{itemize}

Given that, after all, optimization may only result in constant-factor improvements, we suggest to optimize $t$ and $u$ experimentally, {\sl e.g.} using the open source program \btrsync\ developed by the authors ({\sl cf.} section  \ref{program}).

\subsection{How to Stop a Probabilistic Washing Machine?} We now combine both optimizations and assume that $\ell$ laundry rounds are necessary for completing some reconciliation task using a shortened $p$. Unlike section \ref{basic}, confirming correct protocol termination is now non-trivial.\smallskip

If the round failure probability\footnote{{\sl i.e.} that probability that a round resulted in an $(a',b')\neq (a,b)$ satisfying equation (\ref{eqab}).} is some function $\upsilon(w)$ (that we did not estimate) and if $w$ is kept small (for efficiency reasons), the probability $\left(1-\upsilon(w)\right)^{\ell}$ that the protocol will properly terminate may dangerously drift away from one.\smallskip

If $v$ of $\ell+v$ rounds failed, Oscar needs to solve a problem called {\sl Chinese Remaindering With Errors}:\smallskip

\begin{problem}{\sl (Chinese Remaindering With Errors: CRWE).} Given as input integers $v$, $B$ and $\ell+v$ points $(s_1,p_1),\ldots,(s_{\ell+v},p_{\ell+v})\in \mathbb{N}^2$ where the $p_i$'s are coprime, output all numbers $0 \leq s < B$ such that $s \equiv s_i \bmod p_i$ for at least $v$ values of $i$.
\end{problem}

We refer the reader to \cite{phong} for more on this problem, which is beyond the scope of this article and note that Boneh \cite{boneh} provides a polynomial-time algorithm for solving the {\sc crwe} problem under certain conditions satisfied by our setting.\smallskip

To detect that reconciliation succeeded, Neil will send to Oscar $\mbox{{\tt Hash}}(\mathfrak{F}')$ as soon as the interaction starts. As long as Oscar's {\sc crwe} resolution does not result in a state matching $\mbox{{\tt Hash}}(\mathfrak{F}')$, the parties will continue the interaction.

\section{Asymptotic Computational Complexity}

Let $\mu(k)$ be the time required to multiply two $k$-bit numbers\footnote{We suppose that $\forall k,k', \mu(k+k') \ge \mu(k) + \mu(k')$}.
The division and the modular reduction of two $k$-bit numbers modulo a $k$-bit number costs $\Oapp(\mu(k))$ \cite{burnikel1998fast}.
For naive algorithms $\mu(k) = O(k^2)$, but using {\sc fft} multiplication strategies \cite{schonhage1971schnelle}, $\mu(k) = \Oapp(k)$.
{\sc fft} is experimentally faster than other methods (na\"{i}ve or Karatsuba) for $k \sim 64\cdot 10^6$ and on. For such sizes, in packages such as {\sf gmp}, division and modular reduction also run in $\Oapp(\mu(k))$.

Since $p \sim 2^{ut}$, here are the costs:

\begin{figure}
  \begin{tabularx}{\textwidth}{|l|X|p{1.2cm}p{2.3cm}|p{1.2cm}p{2.3cm}|}\hline
{\bf \hfill Entity \hfill \null} & {\bf \hfill Computation \hfill \null} & \multicolumn{4}{c|}{{\bf Complexity expressed in $\Oapp$ of}} \\\hline\hline
Both  & redundancies $c$ and $c'$                                        & $n \cdot \mu(u t)$  & na\"{i}ve product & $n u t$  & using {\sc fft}             \\\hline
Oscar & $s = c' / c \bmod p$                                             & $\mu(u t)$    & na\"{i}ve inversion& $u t$    & using {\sc fft}               \\\hline
Oscar & $a,b$ such that $s = a / b \bmod p$                              & $(u t)^2$   & na\"{i}ve ext. {\sc gcd} & $\mu(u t)$ &  using \cite{pan2004rational,wang2003acceleration} \\\hline
Both  & factorization of $a$ (resp. $b$) by modular reductions           & $n \cdot \mu(u t)$  & na\"{i}ve reduction & $n u t$  & using {\sc fft}             \\\hline
      & {\bf Overwhelming complexity:}                                   & \multicolumn{2}{c|}{$\max((u t)^2,n \cdot \mu(ut))$ }    & \multicolumn{2}{c|}{$n u t$}              \\\hline
  \end{tabularx}
  \caption{Global Protocol Complexity.}
  \label{tab:workload}
\end{figure}


\subsection{Improvements}

It is possible to improve the complexity of the computation of the redundancy and the factorization to $\Oapp(n/t \mu(u t)$, $\Oapp(n u)$ with FFT~\cite{schonhage1971schnelle}.
To simplify the explanations, let us suppose $t$ is a power of $2$: $t=2^\tau$, and $t$ divides $n$.

The idea is the following: we group $h_i$ by group of $t$ elements and we compute the product of each of these groups in $\mathbb{N}$.
\[ H_j = \prod_{i=j t}^{j t + t - 1} h_i. \]
Each of these products can be computed in $\Oapp(\mu(u t))$ using a standard method of product tree, depicted in Algorithm~\ref{alg:prod-tree} (for $j=0$) and in Figure~\ref{fig:prod-tree}.
And all these $n / t$ products can be computed in $\Oapp(n/t \mu(u t))$.
Then, one can compute $c$ by multiplying these products $H_j$ together, modulo $p$, which costs $\Oapp(n/t \mu(u t))$.

\begin{figure}[t]
\centering
\centerline{
\begin{turn}{90}
\begin{tikzpicture}[level/.style={sibling distance=60mm/#1}]
\node (z){$\displaystyle \pi=\pi_1=\prod_{i=0}^{t-1} h_i$}
  child {node (a) {$\displaystyle \pi_2=\prod_{i=0}^{t/2-1} h_i$}
    child {node (b) {$\displaystyle \pi_4=\prod_{i=0}^{t/4-1} h_i$}
      child {node {$\vdots$}
        child {node (d) {$\displaystyle \pi_t=h_0$}}
        child {node (e) {$\displaystyle \pi_{t+1}=h_1$}}
      }
      child {node {$\vdots$}}
    }
    child {node (g) {$\displaystyle \pi_5=\prod_{i=t/4}^{t/2-1} h_i$}
      child {node {$\vdots$}}
      child {node {$\vdots$}}
    }
  }
  child {node (j) {$\displaystyle \pi_3=\prod_{i=t/2}^{t-1} h_i$}
    child {node (k) {$\displaystyle \pi_6=\prod_{i=t/2}^{3t/4-1} h_i$}
      child {node {$\vdots$}}
      child {node {$\vdots$}}
    }
    child {node (l) {$\displaystyle \pi_7=\prod_{i=3t/4}^{t-1} h_i$}
      child {node {$\vdots$}}
      child {node (c){$\vdots$}
        child {node (o) {$\displaystyle h_{t-2}$}}
        child {node (p) {$\displaystyle \pi_{2t-1}=h_{t-1}$}
%
%
          child [grow=right] {node (qe) {} edge from parent[draw=none]
            child [grow=right] {node (q) {$2^\tau \mu(u) \le \mu(u t)$} edge from parent[draw=none]
            child [grow=up] {node (r) {$\vdots$} edge from parent[draw=none]
            child [grow=up] {node (s) {$4 \mu(u t/4) \le \mu(u t)$} edge from parent[draw=none]
            child [grow=up] {node (t) {$2 \mu(u t/2) \le \mu(u t)$} edge from parent[draw=none]
            child [grow=up] {node (u) {$\mu(u t)$} edge from parent[draw=none]}
          }}}
          child [grow=down] {node (v) {$\tau \mu(u t) = \Oapp(\mu(u t))$}edge from parent[draw=none]}
            }
          }
        }
    }
  }
};
%\path (o) -- (e) node (x) [midway] {$\cdots$}
%  child [grow=down] {
%    node (y) {$O\left(\displaystyle\sum_{i = 0}^k 2^i \cdot \frac{n}{2^i}\right)$}
%    edge from parent[draw=none]
%  };
\path (q) -- (r) node [midway] {+};
\path (s) -- (r) node [midway] {+};
\path (s) -- (t) node [midway] {+};
\path (s) -- (l) node [midway] {$\displaystyle \longrightarrow$};
\path (t) -- (u) node [midway] {+};
\path (z) -- (u) node [midway] {$\displaystyle \longrightarrow$};
\path (j) -- (t) node [midway] {$\displaystyle \longrightarrow$};
\path (p) -- (q) node [midway] {$\displaystyle \longrightarrow$};
%\path (y) -- (x) node [midway] {$\Downarrow$};
%\path (v) -- (y)
%  node (w) [midway] {$\tau \mu(u t) = \Oapp(\mu(u t))$};
\path (q) -- (v) node [midway] {$\displaystyle \le$};
%\path (e) -- (x) node [midway] {+};
%\path (o) -- (x) node [midway] {+};
%\path (y) -- (w) node [midway] {$\displaystyle \longrightarrow$};
%\path (v) -- (w) node [midway] {$\Leftrightarrow$};
%\path (r) -- (c) node [midway] {$\cdots$};
\end{tikzpicture}
\end{turn}
}
\caption{Product tree}\label{fig:prod-tree}
\end{figure}



\begin{algorithm}
\newcommand{\vstart}{\ensuremath{\mathrm{start}}}
\newcommand{\vmid}{\ensuremath{\mathrm{mid}}}
\newcommand{\vend}{\ensuremath{\mathrm{end}}}
\begin{algorithmic}[1]
\Require{a table $h$ such that $h[i] = h_i$}
\Ensure{$\pi = \pi_1 = \prod_0^{t-1} h_i$, and $\pi[i] = \pi_i$ for $i \in \{1,\dots,2t-1\}$ as in Figure~\ref{fig:prod-tree}}
\State $\pi \gets $ array of size $t$
\Function{prodTree}{$i$,$\vstart$,$\vend$}
  \If{$\vstart = \vend$}
    \State \Return $1$
  \ElsIf{$\vstart+1 = \vend$}
    \State \Return $h[\vstart]$
  \Else
    \State $\vmid \gets \lfloor (\vstart+\vend)/2 \rfloor$
    \State $\pi[i] \gets $\Call{prodTree}{$2\times i$,$\vstart$,$\vmid$}
    \State $\pi[i+1] \gets $\Call{prodTree}{$2\times i+1$,$\vstart$,$\vmid$}
    \State \Return  $\times$ \Call{prodTree}{$\vmid$,$\vend$}
  \EndIf
\EndFunction
\State $\pi[1] \gets $\Call{prodTree}{$1,0,t$}
\end{algorithmic}
\caption{Product Tree Algorithm}\label{alg:prod-tree}
\end{algorithm}

The same technique applies for the factorization, but this time, we have to be a bit more careful.
After computing the tree product, we can compute the residues of $a$ (or $b$) modulo $H_j$, then we can compute the residues of these new elements modulo the two children of $H_j$ in the product tree ($\prod_{i=j t}^{j t + t/2 - 1} h_i$ and $\prod_{i=j t}^{j t + t/2 - 1} h_i$), and then compute the residues of these two new values modulo the children of the previous children, and so on.
Intuitively, we go down the product tree doing modulo reduction.
At the end ({\sl i.e.}, at the leaves), we obtain the residues of $a$ modulo each of the $h_i$.
This is illustrated in Algorithm~\ref{fig:div-prod-tree} and Figure~\ref{fig:div-prod-tree} (for $j=1$).
Complexity is $\Oapp(\mu(u t))$, for each $j$.
So the total complexity is $\Oapp(n/t \Oapp(\mu(u t))$.

\begin{figure}[t]
\centering
\centerline{
\begin{turn}{90}
\begin{tikzpicture}[level/.style={sibling distance=60mm/#1}]
\node (z){$\displaystyle a \bmod \pi_1$}
  child {node (a) {$\displaystyle a \bmod \pi_2$}
    child {node (b) {$\displaystyle a \bmod \pi_3$}
      child {node {$\vdots$}
        child {node (d) {$\displaystyle a \bmod h_{0}$}}
        child {node (e) {$\displaystyle a \bmod h_{1}$}}
      }
      child {node {$\vdots$}}
    }
    child {node (g) {$\displaystyle a \bmod \pi_5$}
      child {node {$\vdots$}}
      child {node {$\vdots$}}
    }
  }
  child {node (j) {$\displaystyle a \bmod \pi_3$}
    child {node (k) {$\displaystyle a \bmod \pi_6$}
      child {node {$\vdots$}}
      child {node {$\vdots$}}
    }
    child {node (l) {$\displaystyle a \bmod \pi_7$}
      child {node {$\vdots$}}
      child {node (c){$\vdots$}
        child {node (o) {$ $}}
        child {node (p) {$\displaystyle a \bmod h_{t-1}$}
%
%
          child [grow=right] {node (qe) {} edge from parent[draw=none]
            child [grow=right] {node (q) {$2^\tau O(\mu(u)) = O(\mu(u t))$} edge from parent[draw=none]
            child [grow=up] {node (r) {$\vdots$} edge from parent[draw=none]
            child [grow=up] {node (s) {$4 O(\mu(u t/4)) = O(\mu(u t))$} edge from parent[draw=none]
            child [grow=up] {node (t) {$2 O(\mu(u t/2)) = O(\mu(u t))$} edge from parent[draw=none]
            child [grow=up] {node (u) {$O(\mu(u t))$} edge from parent[draw=none]}
          }}}
          child [grow=down] {node (v) {$\tau O(\mu(u t)) = \Oapp(\mu(u t))$}edge from parent[draw=none]}
            }
          }
        }
    }
  }
};
%\path (o) -- (e) node (x) [midway] {$\cdots$}
%  child [grow=down] {
%    node (y) {$O\left(\displaystyle\sum_{i = 0}^k 2^i \cdot \frac{n}{2^i}\right)$}
%    edge from parent[draw=none]
%  };
\path (q) -- (r) node [midway] {+};
\path (s) -- (r) node [midway] {+};
\path (s) -- (t) node [midway] {+};
\path (s) -- (l) node [midway] {$\displaystyle \longrightarrow$};
\path (t) -- (u) node [midway] {+};
\path (z) -- (u) node [midway] {$\displaystyle \longrightarrow$};
\path (j) -- (t) node [midway] {$\displaystyle \longrightarrow$};
\path (p) -- (q) node [midway] {$\displaystyle \longrightarrow$};
%\path (y) -- (x) node [midway] {$\Downarrow$};
%\path (v) -- (y)
%  node (w) [midway] {$\tau \mu(u t) = \Oapp(\mu(u t))$};
\path (q) -- (v) node [midway] {$\displaystyle =$};
%\path (e) -- (x) node [midway] {+};
%\path (o) -- (x) node [midway] {+};
%\path (y) -- (w) node [midway] {$\displaystyle \longrightarrow$};
%\path (v) -- (w) node [midway] {$\Leftrightarrow$};
%\path (r) -- (c) node [midway] {$\cdots$};
\end{tikzpicture}
\end{turn}}
\caption{Division from product tree}\label{fig:div-prod-tree}
\end{figure}


\begin{algorithm}
\newcommand{\vstart}{\ensuremath{\mathrm{start}}}
\newcommand{\vmid}{\ensuremath{\mathrm{mid}}}
\newcommand{\vend}{\ensuremath{\mathrm{end}}}
\begin{algorithmic}[1]
\Require{$a$ an integer, $\pi$ the product tree from Algorithm~\ref{alg:prod-tree}}
\Ensure{$A_i = A[i] = a \bmod \pi_i$ for $i \in \{1,\dots,2t-1\}$, computed as in Figure~\ref{alg:div-prod-tree}}
\State $A \gets $ array of size $t$
\Function{modTree}{$i$}
  \If{$i < 2t$}
    \State $A[i] \gets A[\lfloor i/2 \rfloor] \bmod \pi[i]$
    \State \Call{modTree}{$2 \times i$}
    \State \Call{modTree}{$2 \times i+1$}
  \EndIf
\EndFunction
\State $A[1] \gets a \bmod \pi[1]$
\State \Call{modTree}{$2$}
\State \Call{modTree}{$3$}
\end{algorithmic}
\caption{Division using product tree}\label{alg:div-prod-tree}
\end{algorithm}

\section{Optimizing Parameters}

The proposed process lends itself to a final fine-tuning. We list here some of the proposed research directions that could be investigated to that end:

\subsection{Using a Smooth $p$}

Comme explique dans un ancien email, je pense que l'on devrait utiliser un produit de petits nombres premiers au lieu d'un grand nombre premier $p$. Des l'instant que ces petits nombres premiers sont plus grands que les hashes, cela fonctionne. L'interêt est que l'on peut travailler modulo ces "petits nombres premiers" avec le CRT. Et en plus, la generation de ce modulo $p$ (pas premier) est beaucoup plus rapide.

Faster hashing into the primes using $h-h \bmod \pi + k \pi$ where $\pi$ is a product of small primes

\section{Implementation}
\label{program}

To illustrate the concept, the authors has coded and evaluated the proof of concept described in this section.\smallskip

The executable and source codes of the program, called {\sf btrsync}, can be downloaded from: \url{https://github.com/RobinMorisset/Btrsync}.\smallskip

The synchronisation is unidirectional (clearer). The program consist in two subprograms: a bash script and a python script:

\subsection{The Bash Script}

A bash script runs a python script (describe below) on the two computers to be synchronized. If the computer is not the one running the bash script, the python script is executed through ssh. The bash scripts also creates two pipes: one from Neil stdin to Oscar stdout and one from Oscar stdin to Neil stdout. Data exchanged during the protocol transits {\sl via} these two pipes.

\subsection{The Python Script}

The python script uses gmp which implements all the number theory operations required by Oscar and Neil, and does the actual synchronization. This script works in two phases:

\subsubsection{Finding Different Files}

\begin{enumerate}
\item Compute the hashes of all files concatenated with their paths, type (folder/file), and permissions (not supported yet).
\item Implement the protocol proposed in Section \ref{} [add here a reference to the appropriate section in the paper] with input data coming from stdin and output data going to stdout.
\end{enumerate}

More precisely:
\begin{itemize}
\item Oscar sends it product of hashes modulo a first prime number $p_1$.
\item Neil receives the product, divides by its own product of hashes, reconstructs the fraction modulo $p_1$ [can we elaborate more on what happens here? which functions in GMP are used to do the reconstruction?] and checks if he can factor the denominator using his hashes base. If he can, he stops and sends the numerator and the list of tuples (path, type, hash of content of the file) corresponding to the denominator's factors. Otherwise he sends "None" [is this the ASCII string "None"? if not what does he send precisely?].
\item If Neil sent "None", Oscar computes the product of hashes modulo another prime $p_2$, sends it... CRT mechanism... [can we elaborate more on what happens here? which functions in GMP are used to do the CRT?]
\item If Neil sent the numerator and a list of tuples, then Oscar factors the numerator over his own hash values. Now each party (Neil, Oscar) knows precisely the list of files (path + type + hash of content) that differs from the over party.
\end{itemize}

[please structure the following:]\smallskip

2. synchronize all the stuff [this is not an expression we can use in a paper...]. This part is not completely optimized.\smallskip

We just remove all folders Oscar should not have and create new folders.\smallskip

Then we remove all files Oscar should not have and synchronize using rsync the last files.\smallskip

We could check for move (since we have the list of hash of contents of files) and do moves locally.\smallskip

We can even try to detect moves of complete subtrees...\smallskip

\subsection{Move Resolution Algorithm}

To reproduce the structure of Oscar on Neil, we have a list of file moves to
apply. Sadly, it is not straightforward to apply the moves, because, if we take
a file to move, its destination might be blocked, either because a file already
exists (we want to move $a$ to $b$, $b$ already exists), or because a folder
cannot be created (we want to move $a$ to $b/c$, $b$ already exists but is a
file and not a folder). Note that for a move operation $a \rightarrow b$, there
is at most one file blocking the location $b$: we will call it the
\emph{blocker}.

If the blocker is not present on Oscar, then we can just delete. However, if it
exists, then we might need to move it somewhere else before we solve the move we
are interested in. This move itself might have a blocker, and so on. It seems
that we just need to continue until we reach a move which has no blocker or
where the blocker can be deleted, but we can get caught in a cycle: if we must
move $a$ to $b$, $b$ to $c$ and $c$ to $a$, then we will not be able to perform
anything without using a temporary location.

How can we perform the moves? A simple way would be to move each file to a
unique temporary location and then rearrange files to our liking: however, this
performs many unnecessary moves and will result in problems if the program is
interrupted. We can do something more clever so by performing a decomposition in
strongly connected components of the \emph{move graph} (with one vertex per file
and one edge per move operation going from to the file to its blocker or to its
destination if no blocker exists). The computation of the SCC decomposition is
simplified by the observation that because two files being moved to the same
destination must be equal, we can only keep one arbitrary in-edge per node, and
look at the graph pruned in this fashion: its nodes have in-degree at most one,
so the strongly connected components are either single nodes or cycles. Once the
SCC decomposition is known, the moves can be applied by applying each SCC in a
bottom-up fashion, an SCC's moves being solved either trivially (for single
files) or using one intermediate location (for cycles).

The detailed algorithm is implemented as two mutually recursive functions and
presented as Algorithm~\ref{alg:moves}.

% TODO check this algo
\begin{algorithm}
  \caption{Perform moves}
  \label{alg:moves}
  \begin{algorithmic}[1]
    \Require{$D$ is a dictionary where $D[f]$ denotes the intended destinations of $f$}
    \Statex
    \Let{$M$}{[]}
    \Let{$T$}{[]}
    \For{$f$ in $D$'s keys}
      \Let{$M[f]$}{not\_done}
    \EndFor
    \Function{unblock\_copy}{$f, t$}
      \If{$t$ is blocked by some $b$}
        \If{$b$ is not in $D$'s keys}
          \State unlink($b$) \Comment{We don't need $b$}
        \Else
          \State \Call{resolve}{$b$} \Comment{Take care of $b$ and make it go away}
        \EndIf
      \EndIf
      \If{$T[f]$ was set}
        \Let{$f$}{$T[f]$}
      \EndIf
      \State copy($f$, $d$)
    \EndFunction
    \Function{resolve}{$f$}
      \If{$M[f] =$ done}
        \State \Return \Comment{Already managed by another in-edge}
      \EndIf
      \If{$M[f] =$ doing}
        \Let{$T[f]$}{mktemp()}
        \State move($f$, $T[f]$)
        \Let{$M[f]$}{done}
        \State \Return \Comment{We found a loop, moved $f$ out of the way}
      \EndIf
      \Let{$M[f]$}{doing}
      \For{$d \in D[f]$}
        \If{$d \neq f$}
          \State unblock\_copy($f$, $d$) \Comment{Perform all the moves}
        \EndIf
      \EndFor
      \If{$f \notin D[f]$ and $T[f]$ was not set}
        \State unlink($f$)
      \EndIf
      \If{$T[f]$ was set}
        \State unlink($T[f]$)
      \EndIf
    \EndFunction

    \For{$f$ in $D$'s keys}
      \State \Call{resolve}{$f$}
    \EndFor
  \end{algorithmic}
\end{algorithm}

An optimization implemented by \btrsync over the algorithm described here is to
move files instead of copying them and then removing the original file, because
moves are faster than copies on most filesystems as they don't need to copy the
file contents.

\subsection{Experimental Comparison to \rsync}

To demonstrate the benefits of our approach, we compared our \btrsync
implementation to the standard \rsync on the following datasets:

\begin{description}
  \item[\texttt{synthetic}] A directory containing 1000 very small files
    containing the numbers from 1 to 1000.
  \item[\texttt{synthetic\_shuffled}] The result of applying a few operations to
    \texttt{synthetic}: 10 files were deleted, 10 files were renamed, and the
    contents of 10 files was changed.
  \item[\texttt{source}] A snapshot of the \btrsync source tree.
  \item[\texttt{source\_moved}] The result of renaming a big folder (several
    hundred of kilobytes) in \texttt{source}.
  \item[\texttt{firefox-13.0}] The source archive of Mozilla Firefox 13.0.
  \item[\texttt{firefox-13.0.1}] The source archive of Mozilla Firefox 13.0.1.
  \item[\texttt{empty}] An empty folder.
\end{description}

We performed the measurements with \rsync version 3.0.9 (used both
as the standalone \rsync and for the underlying call to \rsync in
\btrsync). The standalone \rsync was given the \texttt{--delete} flag to delete
existing files on Oscar which do not exist on Neil. We passed the \texttt{-I}
flag to ensure that \rsync did not cheat by looking at the file modification
times (which \btrsync does not do), and \texttt{--chmod="a=rx,u+w"} as an
attempt to disable the transfer of file permissions (which \btrsync does not
transfer). Although these settings ensure that \rsync does not need to transfer
the permissions, verbose logging suggests that it transfers them anyway, so
\rsync must lose a few bytes per file as compared to \btrsync for this reason.

Bandwidth accounting was performed with the \texttt{-v} flag for \rsync
invocations (which report the number of sent and received bytes), and by
counting the amount of data transmitted for \btrsync's own negociations. The
experiment was performed between two remote hosts on a high-speed link. The time
measurements account both for CPU time and for transfer time and are just given
as a general indication.

Results are given in table~\ref{tab:results}. The timing results show that
\btrsync is slower than \rsync, especially when the number of files is high
(i.e., the synthetic datasets). The bandwidth results, however, are more
satisfactory. It is true that the trivial datasets where either Oscar or Neil
have no data, \rsync outperforms \btrsync: this is especially clear in the case
where Neil has no data: \rsync immediately notices that there is nothing to
transfer, whereas \btrsync transfers information to determine the symmetric
difference. On the non-trivial datasets, however, \btrsync outperforms \rsync.
This is the case on the synthetic datasets, where \btrsync does not have to
transfer information about all the files which were not modified, and even more
so on the case where there are no modifications at all. On the Firefox source
code dataset, \btrsync saves a very small amount of bandwidth, presumably
because of unmodified files. For the \btrsync source code dataset, we notice
that \btrsync, unlike \rsync, was able to detect the move and to avoid
retransferring the moved folder.

\begin{figure}
  \begin{tabularx}{\textwidth}{|X|X||c c|c c|c c|c c|}
    \hline
    \multicolumn{2}{|c||}{\bf Entities and Datasets} & \multicolumn{6}{c|}{\bf Transmission (Bytes)} & \multicolumn{2}{c|}{\bf Time (s)} \\
    \hline {\bf \hfill Neil's $\mathfrak{F}'$ \hfill \null} & {\bf \hfill Oscar's $\mathfrak{F}$ \hfill \null}
    & {\bf TX$_r$} & {\bf RX$_r$} & {\bf TX$_b$} & {\bf RX$_b$} & {\bf abs} & {\bf rel} & {\bf t$_r$} & {\bf t$_b$} \\\hline 
    &&&&&&&&&\\[-1em]
\texttt{source} & \texttt{empty} & 1613 & 778353 & 1846 & 788357 & 10237 & +2 \% & 0.2 & 7.7 \\
\texttt{empty} & \texttt{source} & 11 & 29 & 12436 & 6120 & 18516 & +46305 \% & 0.1 & 5.5 \\
\texttt{empty} & \texttt{empty} & 11 & 29 & 19 & 28 & 7 & +32 \% & 0.1 & 0.3 \\
\texttt{synthetic} & \texttt{synthetic\_shuffled} & 24891 & 51019 & 3638 & 4147 & -68125 & -57 \% & 0.2 & 26.8 \\
\texttt{synthetic\_shuffled} & \texttt{synthetic} & 24701 & 50625 & 3443 & 3477 & -68406 & -58 \% & 0.2 & 26.6 \\
\texttt{synthetic} & \texttt{synthetic} & 25011 & 50918 & 327 & 28 & -75574 & -67 \% & 0.1 & 25.7 \\
\texttt{firefox-13.0.1} & \texttt{firefox-13.0} & 90598 & 28003573 & 80895 & 27995969 & -17307 & +0 \% & 2.6 & 4.2 \\
\texttt{source\_moved} & \texttt{source} & 2456 & 694003 & 1603 & 1974 & -692882 & -99 \% & 0.2 & 2.5 \\
\hline

  \end{tabularx}
  \caption{Experimental results. The two first columns indicate the datasets,
    synchronization is performed \emph{from} Neil \emph{to} Oscar. RX and TX are
    received and sent byte counts, $r$ and $b$ are \rsync and \btrsync, we also
    provide the absolute difference in exchanged data (positive when \btrsync
    transfers more data than \rsync) and the relative amount of data sent by
    \btrsync compared to \rsync (over $100\%$ when \btrsync transfers more data
    than $\rsync$). The last two columns show timing results.}
  \label{tab:results}
\end{figure}

\section{Haskell Implementation}

(Antoine: je range ça dans sa propre section, mais j'imagine qu'on va le couper
ou juste dire que ça existe.)

\subsection{Program Structure}

A proof-of-concept called Btrsync has been implemented in Haskell and is
available at https://github.com/RobinMorisset/Btrsync.

It is intended to work as a drop-in replacement of rsync for directories, taking
as arguments two (possibly remote) directories. It launches instances of itself
on each of these machines (by ssh), playing respectively Neil and Oscar's roles.

Communication between Neil and Oscar is handled by the original instance, that
links each agent standard output to the standard input of the other.

Neil does almost all computations, while Oscar send him the needed
informations and run the effective transfer of files when the
computations are done. Btrsync uses rsync to synchronize single files,
because it's algorithm to detect changes in a files is very good.

\subsection{Time Measurements}

Because of difficulties in linking with the GMP library the code is
significantly slower than it could be (especially in the computation of the
primes from the hashes).

TODO: benchmarks with time + bandwidth (our only benefit ..)

\section{Conclusion and Further Improvements}

\comm{Recopier de l'intro et adapter}

We strongly encourage the developer community to continue improving our open source and public-domain software (that we called \btrsync).

\section{Acknowledgment}

The authors acknowledge Guillain Potron for his early involvement in this research work.\smallskip

\section{ToDo}

todo: Fix euclidean to Euclidean in reference 5.\smallskip

todo: Merge two reference files rsynch and wagner.\smallskip

\nocite{rsync}
\nocite{wagner}

\bibliographystyle{splncs03}
\bibliography{btrsync}

\begin{thebibliography}{30}

\bibitem{boneh} D. Boneh, {\sl Finding Smooth Integers in Short Intervals Using CRT Decoding}, Proceedings of the 32-nd Annual ACM Symposium on Theory of Computing, 2000, pp. 265--272.

\bibitem{phong} D. Bleichenbacher and Ph. Nguyen, Advances in Cryptology -- Proceedings of Eurocrypt'00, vol. 1807 of Lecture Notes in Computer Science, Springer-Verlag, pp. 53--69.

\bibitem{PSRec} Y. Minsky, A. Trachtenberg, {\sl Scalable Set Reconciliation}, 40th Annual Allerton Conference on Communications, Control and Computing, Monticello, IL, October 2002. A full version entitled {\sl Practical Set Reconciliation} can be downloaded from \url{http://ipsit.bu.edu/documents/BUTR2002-01.ps}

\bibitem{Mins1} Y. Minsky, A. Trachtenberg, R. Zippel, {\sl Set reconciliation with nearly optimal communication complexity}. IEEE Transactions on Information Theory, 49(9), 2003, pp. 2213–2218.

\bibitem{Whats} D. Eppstein, M. Goodrich, F. Uyeda, G. Varghese What's the difference?: efficient set reconciliation without prior context
ACM SIGCOMM Computer Communication Review - SIGCOMM '11, 41(4), 2011, pp. 218-229.

\end{thebibliography}

\appendix

\section{Extended Protocol}

\begin{center}
\begin{tabular}{|lcl|}\hline
\multicolumn{3}{|c|}{{\sf First phase during which Neil amasses modular information on the difference~~}} \\\hline
~~{\bf Oscar}                      &                                                      &   {\bf Neil}~\\
                                   &                                                      &start the protocol with $p_1$~\\
                                   &~~{{\LARGE $\stackrel{c_1}{\longrightarrow}$}}~~      &   \\
                                   &                                                      &computes $a,b$ using $p_1$~\\
                                   &                                                      &if $a$ factors properly then go to {\sf Final Phase}\\
                                   &                                                      &~~~~~~else perform the protocol with $p_2$~~\\
                                   &~~{{\LARGE $\stackrel{c_2}{\longrightarrow}$}}~~      &   \\
                                   &                                                      &computes $c \bmod p_1 p_2=\mbox{CRT}_{p_1,p_2}(c_1,c_2)$~~\\
                                   &                                                      &computes $a,b$ using $p_1 p_2$~\\
                                   &                                                      &if $a$ factors properly then go to {\sf Final Phase}\\
                                   &                                                      &~~~~~~else perform the protocol with $p_3$~~\\
                                   &~~{{\LARGE $\stackrel{c_3}{\longrightarrow}$}}~~      &   \\
                                   &                                                      &computes $c \bmod p_1 p_2 p_3=\mbox{CRT}_{p_1,p_2,p_3}(c_1,c_2,c_3)$~~\\
                                   &                                                      &computes $a,b$ using $p_1 p_2 p_3$~\\
                                   &                                                      &if $a$ factors properly then go to {\sf Final Phase}\\
                                   &                                                      &~~~~~~else perform the protocol with $p_4$ ~~\\
                                   &                  $\vdots$                            & \\\hline\hline
\multicolumn{3}{|c|}{{\sf Final Phase~~}} \\\hline
                                   &                                                      & \\
                                   &                                                      &Let $\mathfrak{S}=\{F'_i \mbox{~s.t.~} a \bmod h'_i =0\}$~~\\
                                   &~~{\LARGE $\stackrel{\mathfrak{S},b}{\longleftarrow}$}&\\
                                   ~~deletes files s.t. $b \bmod h_i =0$&                                                      &\\
                                   ~~adds $\mathfrak{S}$ to the disk    &                                                      &\\\hline
\end{tabular}
\end{center}

Note that parties do not need to store the $p_i$'s in full. Indeed, the $p_i$s could be subsequent primes sharing their most significant bits. This reduces storage per prime to a very small additive constant $ \cong \mbox{ln}(p_i) \cong \mbox{ln}(2^{2tu+2}) \cong 1.39(tu+1)$ of about $\log_2(tu)$ bits.

\end{document}
